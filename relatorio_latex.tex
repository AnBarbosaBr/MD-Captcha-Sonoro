\documentclass{article}
\usepackage[utf8]{inputenc}
\title{Relatório do Projeto --- Mineração de Dados}
\author{Primeiro autor, Segundo autor e Terceiro autor}
\date{}

\begin{document}
\maketitle
\section{Introdução}

\textcolor{blue}{Descreva a base de dados que você irá analisar e contextualize a tarefa de
predição que será realizada. Apresente brevemente as ferramentas/bibliotecas que utilizou para análises.}

Nesse relatório serão analisados arquivos de audio, cada qual contendo a gravação de 4 letras e/ou números pertencentes ao conjunto $\{a, b, c, d, h, m, n, x, 6, 7\}$. As gravações foram realizadas através do uso de uma aplicação web disponibilizada pelo professor. No desenvolvimento da aplicação serão utilizados os conjuntos de treinamento e validação disponibilizados pelo professor, mas na avaliação do trabalho haverão gravações às quais os alunos não tiveram acesso, a fim de avaliar se o algoritmo utilizado apresentou boa generalização.

A linguagem escolhida para o trabalho foi o \emph{Python}, e a principal bibliotecas utilizadas para a leitura e tratamento dos dados de audio foi a \emph{librosa}. Também utilizou-se \emph{Pandas}, \emph{Numpy} e outras bibliotecas comuns às aplicações relacionadas à análise de dados.


\section{Análise Exploratória}

\textcolor{blue}{Descreva possíveis problemas na base de dados, transformações feitas, atributos gerados etc.
Tente utilizar gráficos para justificar suas propostas.}
\begin{itemize}
\item Arquivos com menos ou mais de 8 segundos
\item Gravações do Google Tradutor (''sotaque de máquina/estrangeiro``)
\end{itemize}}

\section{Metodologia}

Descreva os algoritmos que utilizará e a metodologia de avaliação que será
considerada.

\section{Resultados}

Apresente tabelas de resultados e gráficos que auxiliem a compreensão dos
mesmos.
Note que seu script de experimento deve deixar claro o código gerador de cada tabela/gráfico.

\section{Comentários Finais}

Descreva os principais resultados obtidos e comente sobre:
(i) dificuldades encontradas;
(ii) ideias que não foram exploradas e razões;
\textbf{Não esquecer de enviar junto ao relatório o script (de preferência R/Python) que executa os experimentos na base de dados e gera os resultados apresentados aqui.}

\end{document}
